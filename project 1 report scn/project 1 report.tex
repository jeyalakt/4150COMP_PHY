\documentclass{article}
\usepackage[utf8]{inputenc}
\usepackage{hyperref}
\usepackage{graphicx}



\title{SOLVING ONE DIMENSIONAL POISSON EQUATION
FYS 4150 – COMPUTATIONAL PHYSICS
}
\author{Jeyalakshmi Thoppe Subramanian }


\usepackage{natbib}
\usepackage{graphicx}

\begin{document}

\maketitle
\url{https://github.com/jeyalakt/4150COMP_PHY.git}
\section{Abstract}
In this Project one dimensional Poisson equation with Drichlet boundary conditions is solved with two algorithms -Gaussian Elimination and LU decomposition . It is observed that Gaussian elimination algorithm consumes less time and hence efficient than the LU decomposition. LU decomposition algorithm was used via pre built library function could not be used  for discretion values above in the order of 10000 while Gaussian elimination algorithm can be used .

\section{Introduction}
Poisson equation describes the electrostatic potential generated by localised charge distribution in electromagnetism  . This study intents  to solve  the Poisson equation using two different algorithms -first,  Gaussian elimination algorithm and second LU decomposition method.
We study how these two algorithms are used to solve the Poisson equation at different precision levels and how Gaussian elimination is more efficient that LU decomposition method.
First, the how the second derivative of  Poisson equation  can be approximated to 3 point formula is presented followed by elaboration on the proposed algorithms .Then the results of the algorithms are presented . Finally, a brief discussion on the results and remarks are provided.
\section{Snippets of algorithms}
\begin{figure}[h!]

\centering

\caption{The Universe}
\label{fig:universe}
\end{figure}

\section{Results}
The results in terms of step size , exact function , numerical solution , relative error are stored in a file for each value of n .
CPU time comparision 
The timings of the algorithms were observed to be of the following order
( Note: timing  includes memory read and write and on the other tasks being processed by CPU at that time  )



\section{Discussion}

Normally we intend to choose large value for n , assuming that it will result in higher precision .But from figure 5 (for gaussian elimination algorithm )we observe that this is true upto certain value of n only . This is due to loss of precision in numerical representation in the computer.A suitable value of n =105 gives results at reasonable cost of CPU time and memory 
Also the LU decomposition algorithm consumes 8*1010  bytes (app. 80 GB) of memory for value n=105   which strongly affects our ability performing the calculation.
So Gaussian elimination method is found more efficient.
\section{Conclusion}
The results show the importance of choosing suitable algorithm before solving a problem numerically within the constraints of time and memory and at reasonably accepted precision. Loss of numerical precision in numerical representation in computer also needs to to be considered.
In this study it is found that Gaussian elimination algorithm is more efficient than LU decomposition .

\section{References} 
\citep{adams1995hitchhiker}

\bibliographystyle{plain}
\bibliography{references}
@{ {Thomas, L.H. },
{ Elliptic Problems in Linear Differential Equations over a Network.
Watson Sci. Comput. Lab Report, Columbia University, New York. },
{1949}
}

@{ {Hjorth-Jensen, M.  },
{Computational Physics - Lecture Notes 2015. University of Oslo },
{2015}
}

@{ {Hjorth-Jensen, M. },
{Computational Physics-documents- linear algorithms. University of Oslo},
{2015}
}
\end{document}

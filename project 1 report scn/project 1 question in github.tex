\documentclass{article}
\usepackage[utf8]{inputenc}
\usepackage{hyperref}
\usepackage{graphicx}


Many important differential equations in  Science can be written as 
linear second-order differential equations

!bt
\begin{equation*}
\frac{d^2y}{dx^2}+k^2(x)y = f(x),
\end{equation*}
!et
where $f$ is normally called the inhomogeneous term and $k^2$ is a real function.

A classical equation from electromagnetism is Poisson's equation.
The electrostatic potential $\Phi$ is generated by a localized charge
distribution $\rho (\mathbf{r})$.   In three dimensions 
it reads

!bt
\begin{equation*}
\nabla^2 \Phi = -4\pi \rho (\mathbf{r}).
\end{equation*}
!et
With a spherically symmetric $\Phi$ and $\rho (\mathbf{r})$  the equations
simplifies to a one-dimensional equation in $r$, namely

!bt
\begin{equation*}
\frac{1}{r^2}\frac{d}{dr}\left(r^2\frac{d\Phi}{dr}\right) = -4\pi \rho(r),
\end{equation*}
!et
which can be rewritten via a substitution $\Phi(r)= \phi(r)/r$ as

!bt
\begin{equation*}
\frac{d^2\phi}{dr^2}= -4\pi r\rho(r).
\end{equation*}
!et
The inhomogeneous term $f$ or source term is given by the charge distribution
$\rho$  multiplied by $r$ and the constant $-4\pi$.

We will rewrite this equation by letting $\phi\rightarrow u$ and 
$r\rightarrow x$. 
The general one-dimensional Poisson equation reads then

!bt
\begin{equation*}
-u''(x) = f(x).
\end{equation*}
!et


=== Project 1 a): === 
In this project we will solve the one-dimensional Poisson equation
with Dirichlet boundary conditions by rewriting it as a set of linear equations.


To be more explicit we will solve the equation

!bt
\begin{equation*}
-u''(x) = f(x), \hspace{0.5cm} x\in(0,1), \hspace{0.5cm} u(0) = u(1) = 0.
\end{equation*}
!et
and we define the discretized approximation  to $u$ as $v_i$  with 
grid points $x_i=ih$   in the interval from $x_0=0$ to $x_{n+1}=1$.
The step length or spacing is defined as $h=1/(n+1)$. 
We have then the boundary conditions $v_0 = v_{n+1} = 0$.
We  approximate the second
derivative of $u$ with

!bt
\begin{equation*}
   -\frac{v_{i+1}+v_{i-1}-2v_i}{h^2} = f_i  \hspace{0.5cm} \mathrm{for} \hspace{0.1cm} i=1,\dots, n,
\end{equation*}
!et
where $f_i=f(x_i)$.
Show that you can rewrite this equation as a linear set of equations of the form

!bt
\begin{equation*}
   \mathbf{A}\mathbf{v} = \tilde{\mathbf{b}},
\end{equation*}
!et
where $\mathbf{A}$ is an $n\times n$  tridiagonal matrix which we rewrite as

!bt
\[
    \mathbf{A} = \begin{bmatrix}
                           2& -1& 0 &\dots   & \dots &0 \\
                           -1 & 2 & -1 &0 &\dots &\dots \\
                           0&-1 &2 & -1 & 0 & \dots \\
                           & \dots   & \dots &\dots   &\dots & \dots \\
                           0&\dots   &  &-1 &2& -1 \\
                           0&\dots    &  & 0  &-1 & 2 \\
                      \end{bmatrix},
\]
!et
and $\tilde{b}_i=h^2f_i$.


In our case we will assume  that the source term is 
$f(x) = 100e^{-10x}$, and keep the same interval and boundary 
conditions. Then the above differential equation
has a closed-form  solution given by $u(x) = 1-(1-e^{-10})x-e^{-10x}$ (convince yourself that this is correct by inserting the
solution in the Poisson equation).  We will compare
our numerical solution with this result in the next exercise. 

=== Project 1 b): ===
 
We can rewrite our matrix $\mathbf{A}$ in terms of one-dimensional vectors $a,b,c$  
of length $1:n$. 
Our linear equation reads

!bt
\[
    \mathbf{A} = \begin{bmatrix}
                           b_1& c_1 & 0 &\dots   & \dots &\dots \\
                           a_1 & b_2 & c_2 &\dots &\dots &\dots \\
                           & a_2 & b_3 & c_3 & \dots & \dots \\
                           & \dots   & \dots &\dots   &\dots & \dots \\
                           &   &  &a_{n-2}  &b_{n-1}& c_{n-1} \\
                           &    &  &   &a_{n-1} & b_n \\
                      \end{bmatrix}\begin{bmatrix}
                           v_1\\
                           v_2\\
                           \dots \\
                          \dots  \\
                          \dots \\
                           v_n\\
                      \end{bmatrix}
  =\begin{bmatrix}
                           \tilde{b}_1\\
                           \tilde{b}_2\\
                           \dots \\
                           \dots \\
                          \dots \\
                           \tilde{b}_n\\
                      \end{bmatrix}.
\]
!et




Note well that we do not include the endpoints since the boundary
conditions are used resulting in a fixed value for $v_i$.  A
tridiagonal matrix is a special form of banded matrix where all the
elements are zero except for those on and immediately above and below
the leading diagonal.  Develop a general algorithm first which does
not assume that we have a matrix with the same elements along the
diagonal and the non-diagonal elements.  The algorithm for solving
this set of equations is rather simple and requires two steps only, a
decomposition and forward substitution and finally a backward
substitution.

Before we proceed with the solution of the differential equation, you should now plan the organization of your data flow. Here you will find it convenient to define vectors that will contain the matrix elements, the solution to the problem and the function $f(x)$ when discretized. You should also plan on to read input data, whether you do this from the command line or from a selected file. We recommend strongly that you use dynamical memory allocation.
An example of a C++ program which reads from the command line various input parameters can "be found here":"https://github.com/CompPhysics/ComputationalPhysicsMSU/blob/master/doc/Projects/2018/Project1/CodeExamples/TridiagonalSimple.cpp"

Your first task is to set up the general algorithm (assuming different
values for the matrix elements) for solving this set of linear
equations.  Find also the precise number of floating point operations
needed to solve the above equations. For the general algorithm you
need to specify the values of the array elements $a$, $b$ and $c$ by
inserting their explicit values.


Then you should code the above algorithm and solve the problem for matrices of the size
$10\times 10$, $100\times 100$ and $1000\times 1000$.  That means that you select $n=10$, $n=100$ and
$n=1000$ grid points. 

Compare your results (make plots) with the closed-form solution for the different number of grid points  in the 
interval $x\in(0,1)$.  The different number of grid points corresponds to different step lengths $h$.


=== Project 1 c): === 

Use thereafter the fact that the matrix has identical matrix elements along the diagonal and identical (but different) values for the non-diagonal elements. Specialize your algorithm to the special case and find the number of floating point operations
for this specific tri-diagonal matrix. Compare the CPU time with the general algorithm from the previous point for matrices up to  $n=10^6$ grid points. 

=== Project 1 d): === 

Compute the relative error  in the data set $i=1,\dots, n$,by setting up

!bt
\[
   \epsilon_i=log_{10}\left(\left|\frac{v_i-u_i}
                 {u_i}\right|\right),
\]
!et
as function of $log_{10}(h)$ for the function values $u_i$ and $v_i$.
For each step length extract the max value of the relative error.  
Try to increase $n$ to $n=10^7$.  Make a table of the results and 
comment your results. You can use either the algorithm from b) or c). 